%%Définir le format du document: papier, taille de police, type de document, etc.

\documentclass[11pt, french]{article}

%%%%%%%%%% Packages externes utilisés %%%%%%%%%%%%%%%%%%%
\usepackage[french]{babel}
\selectlanguage{french}
\usepackage[T1]{fontenc}
\usepackage[utf8]{inputenc}

\usepackage{verbatim}
\usepackage{graphicx}
\usepackage{epstopdf}
\usepackage{macro}
\usepackage{algorithm}
\usepackage{algorithmic}
\usepackage{tikz}
\usepackage{smartdiagram}
\usepackage[utf8]{inputenc}
\usepackage{fourier}
\newcommand{\daywidth}{2.2 cm}
\usetikzlibrary[positioning]
\usetikzlibrary{patterns}
\usepackage[french]{babel}
\usetikzlibrary{arrows,shapes,positioning,shadows,trees}

\tikzset{
  basic/.style  = {draw, text width=2cm, drop shadow, font=\sffamily, rectangle},
  root/.style   = {basic, rounded corners=2pt, thin, align=center,
                   fill=blue!30},
  level 2/.style = {basic, rounded corners=6pt, thin,align=center, fill=blue!40,
                   text width=9em},
  level 3/.style = {basic, thin, align=left, fill=gray!30, text width=7.3em}
}
\usetikzlibrary{arrows,positioning}

%La mise en page du rapport, NE PAS MODIFIER.
\usepackage{geometry}
 \geometry{
 a4paper,
 left=20mm,
 right=20mm,
 top=20mm,
 bottom=20mm,
 }

%%%%%%%%% Le corps du document entre begin et end %%%%%%%%%%%%%%%%%%%
\begin{document}

%Page de garde
%%%%%%%%%%%%%%% Page de garde %%%%%%%%%%%%%%%%%%%

\begin{titlepage}{
    \begin{center}
        \vspace* {25mm}
        {\Large \textbf {Université de Cergy-Pontoise}} \\
        \vspace* {10mm}
        {\Large \textbf {RAPPORT}} \\
        \vspace* {10mm}
        pour le projet d'architecture des Ordinateurs \\
        \textbf {Licence d'Informatique deuxième année} \\
        \vspace* {10mm}

	sur le sujet \\
        \vspace* {10mm}
	{\Huge \textsf{Conception d'un processeur 4 bits}} \\
        \vspace* {10mm}
 	rédigé par \\
        \vspace* {10mm}
        {\Large \textbf {AYAD Ishak, METIDJI Fares}} \\
				\vspace* {10mm}
				\noreffig{images/main.png}{12.82cm}{8.5cm} \\
        \date Mai 2018
        \vspace* {10mm}
	\end{center}
}
\end{titlepage}


%Génération automatique de la table des matières, de la liste des figures et de la liste des tableaux
\tableofcontents
\listoffigures

%Une section "remerciements" pourrait être intéressante. C'est une section non numéroté (avec un * )
\section*{Remerciements}
Avant tout développement de ce projet, il apparaît opportun d'adresser nos remerciements à tous ceux qui nous ont aidés pour la réalisation de ce projet. Nous tenons à remercier en premier lieu Monsieur N.Beausse et J.Lorandel, notre encadrant lors de ce projet, auprès duquel nous avons pu bénéficier d'un grand soutien. Nous remercions également Monsieur M.A.Khelif.

\clearpage
\section{Introduction}

\label{sec:introduction}

\paragraph{Contexte:}
Dans le cadre du module d'architecture des ordinateurs du second Semestre de L2, les étudiants doivent réaliser en binôme un projet avec le logiciel Logisim en réutilisant les éléments appris en cours.
Le projet consiste a réalisation un processeur de 4 bits. Notre binôme est composé de Ayad Ishak et de Metidji Fares, étudiants en L2-I dans le groupe D.

\paragraph{Objet:}
Créer un processeur 4 bits.

\paragraph{Outils de développement:}
Nos outils de développement sont ceux qui nous ont été conseillés par notre enseignant et que nous avons utilisé au cours de ce semestre en TP's d'architecture des ordinateurs et Projet. Nous avons utilisé le logiciel logisim. Nous avons synchronisé notre travail en utilisant le service web d'hébergement GitHub qui nous a beaucoup aidés pour travailler en groupe. Enfin, ce rapport de projet a été rédigé avec LaTex sur le service web Overleaf.

\paragraph{Structure du rapport}:
Notre première partie concernera les spécifications de notre projet, elle contiendra toutes les fonctionnalités du processeur. Ensuite, nous parlerons de la partie réalisation, dans laquelle nous présenterons la conception détaillée de notre processeur et viendra après les extensions de notre processeur. Puis nous aborderons la partie déroulement où le calendrier de notre travail et la répartition des tâches seront présentées, pour arriver ensuite à la conclusion où nous allons donner le bilan de notre projet.
\clearpage
\section{Spécification}
\label{sec:specification}

\paragraph{Chapeau:}Nous avons présenté l'objectif du projet dans la section \ref{sec:introduction}. Dans cette section, nous présentons la spécification de notre processeur 4bits réalisé. Ceci correspond principalement au cahier des charges. Pour cela nous allons décrire chaque élément qui le compose.

\subsection{L'ALU}
\label{sec:ALU}
Une ALU est une Unité Arithmétique et Logique. Elle permet de réaliser des opérations sur des
opérandes présentes à ses entrées. Pour ce projet, le but est de réaliser une ALU capable d'effectuer 8
opérations différentes.

\fig{images/ALU-view.PNG}{6cm}{6cm}{Vu d'ensemble sur l'ALU}{view-alu}

\subsection{Le banc de registre}
\label{sec:banc}
Le registre de 4 bits devra suivre le modèle suivant:

\begin{enumerate}
\item deux sorties de 4 bits pour chacune.
\item une entrée de 4 bits.
\item deux signaux de contrôle de lecture d'un bit chacun.
\item un signal de contrôle d'écriture d'un bit.
\item deux ports d’extension et une horloge.
\end{enumerate}

\fig{images/Bacn-de-registre-view.PNG}{9cm}{5cm}{Vu d'ensemble sur le banc de registre}{view-banc}

\clearpage
\subsection{L'unité d'adressage}
\label{sec:unité adressage}

L'unité d'adressage est uniquement composée de deux registre 4 bits : PC et AD.

\begin{enumerate}
\item PC est les registre d'adresse d'instruction.
\item AD est le registre d'adresse de données.
\end{enumerate}

\fig{images/U-adressage-view.PNG}{9cm}{5cm}{Vu d'ensemble sur l'unité d'adressage}{view-U-adressage}

\subsection{L'unité de contrôle}
\label{sec:unité contrôle}

L'unité de contrôle (UC) c'est l’unité qui contrôle l’exécution des instructions machines par le processeur a pour rôle de placer 
les valeurs des différentes signaux de commande de l’architecture à chaque cycle.

Le contrôle du chemin de données:
\begin{enumerate}
\item l’UAL par 3 signaux de commande 
\item le choix de l’opérande X de l’ALU parmi 4 registres, donc 4 bits de commande 
\item le choix de l’opérande Y de l’ALU parmi 4 registres, donc 4 bits de commande 
\item le choix de la destination du rangement parmi 4 registres, donc 4 bits de commande 
\item le choix d’un type d’accès mémoire (vers la mémoire) : Ecriture ou Lecture et le choix de faire un Fetch de l’instruction, soit 3 nouveaux signaux de commande
\end{enumerate}

\fig{images/U-controle-view.PNG}{7cm}{7cm}{Vu d'ensemble sur l'unité de contrôle}{view-U-controle}
\clearpage
\section{Réalisation}
\label{sec:Réalisation}

\paragraph{Chapeau:} Nous avons présenté les spécifications du projet dans la section \ref{sec:specification}. Dans cette section, nous détaillons la conception de notre processeur et les techniques logique utilisées.

\subsection{L'ALU}
\label{sec:re-ALU}

\paragraph{Composition:} Notre ALU est donc composée de :

\begin{enumerate}
\item deux entrées sur quatre bits.
\item une sortie sur quatre bits.
\item huit opération (dont un FULL-adder et FULL-substractor).
\item un décodeur sur trois bits, trois vers huit.
\end{enumerate}

\paragraph{} Voici le tableau des instructions de l'ALU :

\begin{center}
\begin{tabular} {|p{2cm}|p{3cm}|}
\hline
Opération & Code d'Opération \\
\hline
Addition & 000 \\
\hline
OR & 001 \\
\hline
AND & 010 \\
\hline
NOT & 011 \\
\hline
Soustraction & 100 \\
\hline
NOR & 101 \\
\hline
NAND & 110 \\
\hline
NXOR & 111 \\
\hline
\end{tabular}
\end{center}

\fig{images/ALU-schema.png}{12cm}{11cm}{Circuit de l'ALU}{schema-alu}

\subsection{Le banc de registre}
\label{sec:re-banc}
Le banc de registre est l'unité du CPU qui va stocker provisoirement les données provenant de la mémoire RAM ou de la sortie de l'ALU avent de dirigé soit vers l'entrée A ou l'entrée B de l'ALU pour qu'un calcule soit effectué sur celle-ci.

\paragraph{Composition:} Notre Banc de registre est donc composée de :

\begin{enumerate}
\item une entrée un mot de quatre bits
\item quatre registre de quatre bits
\item chaque registre peut envoyer la donnée qu'il stocke soit vers la sortie X (l'entée A de l'ALU) soit vers la sortie Y (l'entrée B de l'ALU) , on a donc deux bits de sélection par registre , soit huit bits de sélection en sortie de registre.
\end{enumerate}

\fig{images/Banc-de-regidtre-schema.png}{13cm}{13cm}{Circuit du banc de registres}{schema-Banc-de-registre}

\clearpage
\subsection{L'unité d'adressage}
\label{sec:re-unité adressage}
L'unité d'adressage contient deux registre de quatre bits : le PC(Programm counter) et le AD(Adress)
\begin{enumerate}
\item Le PC est le registre qui contient l'adresse de la mémoire d'instruction a exécuté c'est-à-dire l'adresse contenant l'instruction a traité pas le CPU.
\item Le AD est le registre contenant l'adresse de la mémoire de données qui est active c'est-à-dire l'adresse contenant les données traitées en lecture ou en écriture par le CPU.
\end{enumerate}

\paragraph{Composition:} Notre unité d'adressage est donc composée de :

\begin{enumerate}
\item deux entrées de quatre bits vers les deux registres.
\item deux signaux de lecture.
\item deux sortie des deux registres.
\end{enumerate}

\fig{images/U-adressage-schema.png}{13cm}{11cm}{Circuit de l'unité d'adressage}{schema-u-adressage}

\subsection{L'unité de contrôle}
\label{sec:re-unité contrôle}

L'unité de contrôle est l'unité principale du CPU elle vient activer tous les signaux de commande du processeur en fonction 
de l’instruction qui a été chargée depuis la mémoire.  Cette unité contient donc le registre d’instruction qui contient  les  différents champs correspondant aux catégories de commandes du processeur.

\paragraph{les différents signaux:}
\begin{enumerate}
\item la sélection de l’opération réaliser par l'ALU.
\item la sélection des registre dont le contenus sont lus et écrits.
\item les indicateurs d'accès mémoire READ,WRITE et FETCH
\end{enumerate}

L'unité de contrôle incrémente également le contenu du registre PC pour que l'instruction suivante soit lue à chaque cycle d'horloge. Si l'entrée Reset est activée, le registre PC est replacé à sa position BOOT. Les instruction peuvent avoir deux format:  

\paragraph{Format registre:}

\begin{center}
\begin{tabular} {|p{3cm}|p{2cm}|p{1.5cm}|p{1.5cm}|p{2.1cm}|}
\hline
OPCODE (4bits) & RES (2bits) & X (2bits) & Y (2bits) & EXT (2bits) \\
\hline
\end{tabular}
\end{center}

\begin{enumerate}
\item OPCODE: code de l’opération a réalisé.
\item RES: code de sélection du registre ou sera stocker le résultat de l'opération.
\item X: sélection du registre contenant l’opérande X (entrée A de l'ALU)
\item Y: sélection du registre contenant l’opérande Y (entrée B de l'ALU)
\item EXT: extension possible des instruction.
\end{enumerate}

\paragraph{Format immédiat:}

\begin{center}
\begin{tabular} {|p{3cm}|p{2cm}|p{3cm}|p{2.1cm}|}
\hline
OPCODE (4bits) & RES (2bits) & ADRESS (4bits) & EXT (2bits) \\
\hline
\end{tabular}
\end{center}

\begin{enumerate}
\item OPCODE: code de l’opération a réalisé.
\item RES: sélection du registre source pour écrire dans la mémoire de données, ou destination pour lire dans la mémoire de données.
\item ADRESS: sélection de l'adresse mémoire à lire ou écrire selon l’opération effectué.
\item EXT: extension possible des instruction.
\end{enumerate}

\paragraph{Le jeu d'instruction de CPU est le suivant:}

\begin{center}
\begin{tabular} {|p{3.5cm}|p{2cm}|p{1.7cm}|}
\hline
Nom de l'instruction & OPCODE & format \\
\hline
ADD & 0000 & Registre \\
\hline
OR & 0001 & Registre \\
\hline
AND & 0010 & Registre \\
\hline
NOT(B) & 0011 & Registre \\
\hline
LOAD & 0100 & Immédiate \\
\hline
STORE & 1000 & Immédiate \\
\hline
SUB & 0101 & Registre \\
\hline
NOR & 0110 & Registre \\
\hline
NAND & 0111 & Registre \\
\hline
NXOR & 1010 & Registre \\
\hline
\end{tabular}
\end{center}

\paragraph{L'activation des signaux de lecture et écriture dans le banc de registre est pilotée comme suit :}

\begin{center}
\begin{tabular} {|p{1cm}|p{1cm}|p{1cm}|p{1cm}|p{1cm}|p{1cm}|}
\hline
Source & Code & (w,r)A & (w,r)B & (w,r)C & (w,r)D\\
\hline
A & 00 & 1 & 0 & 0 & 0\\
\hline
B & 01 & 0 & 1 & 0 & 0\\
\hline
C & 10 & 0 & 0 & 1 & 0\\
\hline
D & 11 & 0 & 0 & 0 & 1\\
\hline
\end{tabular}
\end{center}

\clearpage
\paragraph{L'activation des signaux de commande de l'ALU,READ,WRITE,FETCH et accès à l'unité d'adressage est pilotée comme suit:}

\begin{center}
\begin{tabular} {|p{2cm}|p{2cm}|p{2.5cm}|p{1cm}|p{1.3cm}|p{2.1cm}|p{2cm}|p{1.3cm}|}
\hline
instruction & OPCODE & Sélection ALU & READ & WRITE & WRITE AD& WRITE PC&FETCH\\
\hline
ADD & 0000 & 000 & 0 & 0 & 0 & 1 & 1 \\
\hline
OR & 0001 & 001 & 0 & 0 & 0 & 1 & 1 \\
\hline
AND & 0010 & 010 & 0 & 0 & 0 & 1 & 1 \\
\hline
NOT(B) & 0011 & 011 & 0 & 0 & 0 & 1 & 1 \\
\hline
LOAD & 0100 & 000 & 1 & 0 & 1 & 1 & 1 \\
\hline
STORE & 1000 & 000 & 0 & 1 & 1 & 1 & 1 \\
\hline
SUB & 0101 & 100 & 0 & 0 & 0 & 1 & 1 \\
\hline
NOR & 0110 & 101 & 0 & 0 & 0 & 1 & 1 \\
\hline
NAND & 0111 & 110 & 0 & 0 & 0 & 1 & 1 \\
\hline
NXOR & 1010 & 111 & 0 & 0 & 0 & 1 & 1 \\
\hline
\end{tabular}
\end{center}

\fig{images/U-controle.png}{15cm}{14cm}{Circuit de l'unité de contrôle}{schema-u-controle}

\clearpage
\subsection{Le CPU}
\label{sec:CPU}

les élément du CPU indiqué dans \ref{sec:re-ALU}, \ref{sec:re-banc}, \ref{sec:re-unité adressage}, \ref{sec:re-unité contrôle} serrant lié entre eux de la manière suivante : 

\fig{images/cpu.PNG}{15cm}{14cm}{Schéma du CPU}{schema-cpu}

\paragraph{} L'unité de contrôle va lire les instruction dans la ROM. Elle va les décoder soit au format immédiat soit au format registre en fonction du code d'opération et active ces différents sortie et elle va alimenté les autres composants.
\clearpage
\section{Extension}
\label{sec:extension}

\paragraph{Chapeau :} Nous avons présenté la réalisation de notre projet dans section\ref{sec:Réalisation}. Dans cette section nous allons expliquer les extension apporté au projet.

\subsection{La soustraction}
\label{sec:soustraction}

La soustraction a été implémenté à l'ALU, et possède son propre code d'instruction pour l'Unité de
contrôle. L'Opération est complétement opérationnelle.

\subsubsection{La soustraction}
\label{sec:sub-soustraction}
\paragraph{table de vérité de la soustraction :}
\begin{center}
\begin{tabular} {|p{0.5cm}|p{0.5cm}|p{0.5cm}|p{1cm}|p{1.7cm}|}
\hline
A & B & C & DIFF & BORROW\\
\hline
0 & 0 & 0 & 0 & 0\\
\hline
0 & 0 & 1 & 1 & 1\\
\hline
0 & 1 & 0 & 1 & 1\\
\hline
0 & 1 & 1 & 0 & 1\\
\hline
1 & 0 & 0 & 1 & 0\\
\hline
1 & 0 & 1 & 0 & 0\\
\hline
1 & 1 & 0 & 0 & 0\\
\hline
1 & 1 & 1 & 1 & 1\\
\hline
\end{tabular}
\end{center}

\subsubsection{Schéma de la soustraction}
\label{sec:schema soustraction}

\fig{images/soustracteur-complet.png}{7cm}{7cm}{Circuit logique de la soustraction}{schema-soustraction}

\subsection{Indicateurs de l’UAL} 
\label{sec:Indicateurs}

\subsubsection{Résultat nul}
\label{sec:nul}

pour vérifier si le résultat est nul ou non on a mis en place un comparateur de quatre bits. En comparant entre le résultat obtenu de l'ALU et une constante nulle on a eu le résultat.
\newpage

\fig{images/comparateur.png}{5cm}{5cm}{Circuit logique du comparateur quatre bits}{schema-comparateur}

\fig{images/implementation-comparateur.PNG}{5cm}{3cm}{implémentation du comparateur quatre bits }{schema-comparateur-impl}

\subsubsection{débordement et signaux d'entrée}
\label{sec:deb}

\paragraph{} l'ajout d’un signal d’indication d’OVF, d’un signal de retenue sortante, d’un signal de résultat négatif dans l’ALU  qui  provoquent  l’allumage  d’une  diode  s’ils  sont  actifs.

\paragraph{} l'ajout signaux d’entrée de l’UAL : enA et enB qui, s’ils sont actifs, autorisent la prise en compte des entrées A et
/ou B. Dans le cas contraire, les entrées sont à 0.

\paragraph{} Relier ces signaux à l’unité de contrôle de manière à ce qu’ils apparaissent dans l’instruction dans le champs 
EXT.

\fig{images//over.png}{6cm}{6cm}{implémentation du comparateur quatre bits et des signaux d'entrées}{schema-comparateur-deb}
\clearpage
\section{Annexes}
\label{sec:annexes}

\paragraph{Chapeau :} Nous avons présenté les extension de notre projet dans section\ref{sec:extension}. Dans cette section nous allons expliquer les différents éléments logique utilisé dans le projet.

\subsection{Full adder 1bit} 
\label{sec:full adder}

\paragraph{} Le full adder 1bit fait une addition de 2bits A et B et d'un bit de retenu en entrée. in en ressort un bit de retenu et un bit de résultat.

\fig{images/full-ader.png}{7cm}{7cm}{Circuit logique du Full adder 1bit}{schema-comparateur1}

\subsection{Décodeur 3bits} 
\label{sec:dec3}

\paragraph{} Le décodeur 3bits prend en entrée un mot de 3bits et active l'une des 8 sorties en fonction du signale d'entrée reçu.

\fig{images/decodeur-3.png}{7cm}{7cm}{Circuit logique du décodeur 3bits}{schema-dec}

\subsection{le décodeur d'instructions} 
\label{sec:deci}

\paragraph{} le décodeur d'instruction est un élément primordiale de l'unité de contrôle il prend en entrée un mot de 4bits (OPCODE) et renvoie des signaux de sélection pour l'ALU et l'unité d'adressage.

\fig{images/decodeur-ins.png}{9cm}{9cm}{Circuit logique du décodeur d'instructions}{schema-deci}

\subsection{registre 12bits} 
\label{sec:reg 12}

\fig{images/reg12.png}{9cm}{9cm}{Circuit logique du registre 12bits}{schema-reg12}


\subsection{CPU} 
\label{sec:CPUen}

\paragraph{} le CPU est donc l'assemblage de l'ALU, le banc de registre, l'unité de contrôle, l'unité d'adressage.

\fig{images/CPU.png}{14cm}{12cm}{Vu d'ensemble sur le CPU}{vu cpu}
\clearpage
\section{Déroulement du projet}
\label{sec:deroulement}

\paragraph{Chapeau} Dans cette section, nous décrivons comment le projet a été réalisé en équipe : la répartition des tâches, la synchronisation du travail entre membres de l'équipe, etc.

\subsection{Répartition des tâches}

\begin{center}
\begin{tabular} {|p{3cm}|p{3cm}|}
\hline
AYAD Ishak & METIDJI Fares\\
\hline
UAL & UAL\\
\hline
Banc de registre & Unité d'adressage\\
\hline
Unité de contrôle & \\
\hline
 Modifications personnelles & \\
\hline

\end{tabular}
\end{center}

\begin{center}
\def\angle{0}
\def\radius{3}
\def\cyclelist{{"orange","blue","red","green"}}
\newcount\cyclecount \cyclecount=-1
\newcount\ind \ind=-1
\begin{tikzpicture}[nodes = {font=\sffamily}]
  \foreach \percent/\name in {
      65.0/AYAD Ishak,
      35.0/METIDJI Fares,
    } {
      \ifx\percent\empty\else
        \global\advance\cyclecount by 1
        \global\advance\ind by 1
        \ifnum3<\cyclecount
          \global\cyclecount=0
          \global\ind=0
        \fi
        \pgfmathparse{\cyclelist[\the\ind]}
        \edef\color{\pgfmathresult}    
        \draw[fill={\color!50},draw={\color}] (0,0) -- (\angle:\radius)
          arc (\angle:\angle+\percent*3.6:\radius) -- cycle;
        \node at (\angle+0.5*\percent*3.6:0.7*\radius) {\percent\,\%};
        \node[pin=\angle+0.5*\percent*3.6:\name]
          at (\angle+0.5*\percent*3.6:\radius) {};
        \pgfmathparse{\angle+\percent*3.6} 
        \xdef\angle{\pgfmathresult}
        \angle
      \fi
    };
\end{tikzpicture}
\end{center}

\subsection{Synchronisation du travail}
\begin{itemize}
\item Pour la réalisation du projet nous avons utilisé le service web d'hébergement et de gestion de développement de logiciels GitHub pour synchroniser nos travaux.Ce dernier nous a permis d’effectuer des commites à un rythme élevé au commencement du projet pour que tous les membre de l'équipe aient accès aux bases ,puis après la répartition des tâches le rythme a diminué.
\item Pour la rédaction de ce document nous avons utilisé la plateforme web Overleaf pour répartir les taches, le rythme de rédaction a été très élevé à la fin du projet. 
\end{itemize}

\subsection{Problèmes rencontrés}
\begin{enumerate}
\item L’implémentation de l’opération soustraction.
\item Réalisation de l'unité d'adressage.
\item La compréhension de l'utilité d'un registre d'adresse de données.
\item La compréhension de certain signaux de sorties de l'UC (EXT).
\item notre Bascule D avait un problème donc nous avons utilisé le registre fourni par Logisim. 
\end{enumerate}

\clearpage
\subsection{Calendrier}

\fig{images/calen.PNG}{18cm}{9cm}{Calendrier}{calendrier}
\clearpage
\section{Conclusion}
\label{sec:conclusion}

\paragraph{} Au final, nous avons donc réalisé une simulation d'une processeur  CPU de quatre bits, en prenant d’effectuer quelque changement d'information entre une machine et sa mémoire.

\paragraph{} D'un point de vue de réalisation, nous avons réalisé le principale du processeur avec tous les éléments et composants avec leur exigence fonctionnelles. Nous avons pu approfondir certains points.

\paragraph{} Grâce a ce projet, on a pu découvrir et étudier de prés les processeurs, mais surtout ce projet nous a permis de manipuler les différents instructions machines permettant de piloter le processeur.



%Références bibliographiques du document
%\bibliographystyle{plain}
%\bibliography{bibliographies}
%\nocite{*}

\end{document}
