\section{Annexes}
\label{sec:annexes}

\paragraph{Chapeau :} Nous avons présenté les extension de notre projet dans section\ref{sec:extension}. Dans cette section nous allons expliquer les différents éléments logique utilisé dans le projet.

\subsection{Full adder 1bit} 
\label{sec:full adder}

\paragraph{} Le full adder 1bit fait une addition de 2bits A et B et d'un bit de retenu en entrée. in en ressort un bit de retenu et un bit de résultat.

\fig{images/full-ader.png}{7cm}{7cm}{Circuit logique du Full adder 1bit}{schema-comparateur1}

\subsection{Décodeur 3bits} 
\label{sec:dec3}

\paragraph{} Le décodeur 3bits prend en entrée un mot de 3bits et active l'une des 8 sorties en fonction du signale d'entrée reçu.

\fig{images/decodeur-3.png}{7cm}{7cm}{Circuit logique du décodeur 3bits}{schema-dec}

\subsection{le décodeur d'instructions} 
\label{sec:deci}

\paragraph{} le décodeur d'instruction est un élément primordiale de l'unité de contrôle il prend en entrée un mot de 4bits (OPCODE) et renvoie des signaux de sélection pour l'ALU et l'unité d'adressage.

\fig{images/decodeur-ins.png}{9cm}{9cm}{Circuit logique du décodeur d'instructions}{schema-deci}

\subsection{registre 12bits} 
\label{sec:reg 12}

\fig{images/reg12.png}{9cm}{9cm}{Circuit logique du registre 12bits}{schema-reg12}


\subsection{CPU} 
\label{sec:CPUen}

\paragraph{} le CPU est donc l'assemblage de l'ALU, le banc de registre, l'unité de contrôle, l'unité d'adressage.

\fig{images/CPU.png}{14cm}{12cm}{Vu d'ensemble sur le CPU}{vu cpu}