\section{Extension}
\label{sec:extension}

\paragraph{Chapeau :} Nous avons présenté la réalisation de notre projet dans section\ref{sec:Réalisation}. Dans cette section nous allons expliquer les extension apporté au projet.

\subsection{La soustraction}
\label{sec:soustraction}

La soustraction a été implémenté à l'ALU, et possède son propre code d'instruction pour l'Unité de
contrôle. L'Opération est complétement opérationnelle.

\subsubsection{La soustraction}
\label{sec:sub-soustraction}
\paragraph{table de vérité de la soustraction :}
\begin{center}
\begin{tabular} {|p{0.5cm}|p{0.5cm}|p{0.5cm}|p{1cm}|p{1.7cm}|}
\hline
A & B & C & DIFF & BORROW\\
\hline
0 & 0 & 0 & 0 & 0\\
\hline
0 & 0 & 1 & 1 & 1\\
\hline
0 & 1 & 0 & 1 & 1\\
\hline
0 & 1 & 1 & 0 & 1\\
\hline
1 & 0 & 0 & 1 & 0\\
\hline
1 & 0 & 1 & 0 & 0\\
\hline
1 & 1 & 0 & 0 & 0\\
\hline
1 & 1 & 1 & 1 & 1\\
\hline
\end{tabular}
\end{center}

\subsubsection{Schéma de la soustraction}
\label{sec:schema soustraction}

\fig{images/soustracteur-complet.png}{7cm}{7cm}{Circuit logique de la soustraction}{schema-soustraction}

\subsection{Indicateurs de l’UAL} 
\label{sec:Indicateurs}

\subsubsection{Résultat nul}
\label{sec:nul}

pour vérifier si le résultat est nul ou non on a mis en place un comparateur de quatre bits. En comparant entre le résultat obtenu de l'ALU et une constante nulle on a eu le résultat.
\newpage

\fig{images/comparateur.png}{5cm}{5cm}{Circuit logique du comparateur quatre bits}{schema-comparateur}

\fig{images/implementation-comparateur.PNG}{5cm}{3cm}{implémentation du comparateur quatre bits }{schema-comparateur-impl}

\subsubsection{débordement et signaux d'entrée}
\label{sec:deb}

\paragraph{} l'ajout d’un signal d’indication d’OVF, d’un signal de retenue sortante, d’un signal de résultat négatif dans l’ALU  qui  provoquent  l’allumage  d’une  diode  s’ils  sont  actifs.

\paragraph{} l'ajout signaux d’entrée de l’UAL : enA et enB qui, s’ils sont actifs, autorisent la prise en compte des entrées A et
/ou B. Dans le cas contraire, les entrées sont à 0.

\paragraph{} Relier ces signaux à l’unité de contrôle de manière à ce qu’ils apparaissent dans l’instruction dans le champs 
EXT.

\fig{images//over.png}{6cm}{6cm}{implémentation du comparateur quatre bits et des signaux d'entrées}{schema-comparateur-deb}