\section{Introduction}

\label{sec:introduction}

\paragraph{Contexte:}
Dans le cadre du module d'architecture des ordinateurs du second Semestre de L2, les étudiants doivent réaliser en binôme un projet avec le logiciel Logisim en réutilisant les éléments appris en cours.
Le projet consiste a réalisation un processeur de 4 bits. Notre binôme est composé de Ayad Ishak et de Metidji Fares, étudiants en L2-I dans le groupe D.

\paragraph{Objet:}
Créer un processeur 4 bits.

\paragraph{Outils de développement:}
Nos outils de développement sont ceux qui nous ont été conseillés par notre enseignant et que nous avons utilisé au cours de ce semestre en TP's d'architecture des ordinateurs et Projet. Nous avons utilisé le logiciel logisim. Nous avons synchronisé notre travail en utilisant le service web d'hébergement GitHub qui nous a beaucoup aidés pour travailler en groupe. Enfin, ce rapport de projet a été rédigé avec LaTex sur le service web Overleaf.

\paragraph{Structure du rapport}:
Notre première partie concernera les spécifications de notre projet, elle contiendra toutes les fonctionnalités du processeur. Ensuite, nous parlerons de la partie réalisation, dans laquelle nous présenterons la conception détaillée de notre processeur et viendra après les extensions de notre processeur. Puis nous aborderons la partie déroulement où le calendrier de notre travail et la répartition des tâches seront présentées, pour arriver ensuite à la conclusion où nous allons donner le bilan de notre projet.